\section*{Ejercicio 2}
\graphicspath{{Figuras/ej_02/}}

Tenemos la ecuación logística con retraso
\begin{equation}
    \frac{dN}{dt} = r N(t) \left[1- \frac{N(t-T)}{K}\right]
\end{equation}
la cual se resolvió numéricamente usando diversos valores de los parámetros. En la Figura \ref{02_ejercicio} se observa las simulaciones obtenidas para tres valores del parámetro $r$: $0.3$, $1.2$ y $2$. En todos los casos $T=1$, $K=10$ y $N(t)=2$ para $-T < t \leq 0$. Para $r$ chicos ($r=0.3$), se observa que el sistema presenta un régimen monótono en donde la población tiende al valor de $K$, como la ecuación logística. Luego, al aumentar $r$, primero se observan oscilaciones amortiguada y al seguir aumentando $r$ pasan a observarse oscilaciones sostenidas. Según \cite{Chule},  para $0<rT<e^{-1}$ el sistema posee un comportamiento monótono. Para $e^{-1} < rT < \dfrac{\pi}{2}$ el sistema presenta oscilaciones amortiguadas y por ultimo para $\dfrac{\pi}{2} < rT$ se obtienen oscilaciones sostenidas.

\begin{figure}[h!]
    \centering
    \includegraphics[width=0.5\textwidth]{Regimenes.pdf}
    \caption{Resolución numérica para la ecuación logística con retraso variando el parámetro $r$. Se observan los regímenes monótono, oscilatorio amortiguado y oscilatorio sostenido para $r$ igual a $0.3$, $1.2$ y $2$, respectivamente. $T=1$, $K=10$ y $N(t)=2$ para $-T < t \leq 0$.}
    \label{02_ejercicio}
\end{figure}

Tenemos una aproximación analítica para cuando $T$ es un poco mayor que que el valor critico $T_c = \nicefrac{\pi}{2r}$, la cual viene dada por
\begin{equation}
    N(t) \approx K + C e^{ \frac{\epsilon t}{1 + \nicefrac{\pi^{2}}{4}} } e^{it \left[ 1- \frac{\epsilon \pi}{2 ( 1 + \nicefrac{\pi^{2}}{4})} \right]}
    \label{02_Ec_Aproximada}
\end{equation}
en donde $T = T_c + \epsilon$ (es decir que la aproximación es valida para el régimen de oscilaciones sostenidas). En la Figura \ref{02_Comp_Analitica} se observa una comparación entre la simulación numérica y la solución aproximada valida para $r=2$ y $\epsilon=0.01$. Se observa que la solución analítica aproximada no logra reproducir correctamente la amplitud de las oscilaciones, pero si es posible estimar el periodo de las mismas. En particular, debido al termino $e^{ \frac{\epsilon t}{1 + \nicefrac{\pi^{2}}{4}} }$ de la Ec. \ref{02_Ec_Aproximada}, la solución aproximada diverge para $t\rightarrow \infty$.

\begin{figure}[h!]
    \centering
    \begin{subfigure}[b]{0.48\textwidth}
        \includegraphics[width=\textwidth]{Comp_Analitica_2.pdf}
        \caption{}
        \label{02_Comp_Analitica}
    \end{subfigure}
    \hfill
    \begin{subfigure}[b]{0.48\textwidth}
        \includegraphics[width=\textwidth]{Amplitud_r=2.0.pdf}
        \caption{}
        \label{02_Picos}
    \end{subfigure}
    \caption{En (a) se observa una comparación entre la simulación numérica y la solución aproximada valida para el régimen de oscilaciones sostenidas para $r=2$ y $\epsilon=0.01$. Se observa que la solución analítica aproximada no logra reproducir correctamente la amplitud de las oscilaciones, pero si es posible estimar el periodo de las mismas. En (b), se observa una simulación y la obtención de los picos y valles de las oscilaciones para el calculo de la amplitud de las oscilaciones y su periodo.}
    \label{02_ejercicio_2}
\end{figure}

Luego se procedió a verificar que la amplitud de las oscilaciones es independiente de las condiciones iniciales y que el periodo de las oscilaciones es independiente de $r$. Para ambos casos, se realizo una simulación variando la condición inicial o $r$ según corresponda y se obtuvieron los picos y valles de las oscilaciones para determinar la amplitud y periodo. Un ejemplo del resultado obtenido puede observarse en la Figura \ref{02_Picos}.

\begin{figure}[h!]
    \centering
    \begin{subfigure}[b]{0.48\textwidth}
        \includegraphics[width=\textwidth]{Amplitud_Barrido.pdf}
        \caption{}
        \label{02_Barrido_CI}
    \end{subfigure}
    \hfill
    \begin{subfigure}[b]{0.48\textwidth}
        \includegraphics[width=\textwidth]{Periodo_Barrido.pdf}
        \caption{}
        \label{02_Barrido_r}
    \end{subfigure}
    \caption{En (a) se observa el valor obtenido de las distintas amplitudes en función de las distintas condiciones iniciales. Se observa que la amplitud presenta pequeñas variaciones del orden de $7\times 10^{-4}$. Por otro lado, como puede observarse en (b), el periodo de las oscilaciones no son invariantes respecto al valor de $r$.}
    \label{02_Resutados_Barridos}
\end{figure}

En la Figura \ref{02_Barrido_CI} se observa el valor obtenido de las distintas amplitudes en función de las distintas condiciones iniciales. Se observa que la amplitud presenta pequeñas variaciones del orden de $7\times 10^{-4}$. Por otro lado, como puede observarse en la Figura \ref{02_Barrido_r}, el periodo de las oscilaciones no son invariantes respecto al valor de $r$.